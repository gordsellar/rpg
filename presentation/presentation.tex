\documentclass{beamer}
\usepackage[utf8]{inputenc}
\usepackage[french]{babel}

\usetheme{Warsaw}


\title{Présentation d'IHME}
\author{SASSOULAS Pierre\\RIVOIRE Claire\\PATTYN Maxime\\GUINGOIN Sylvain}

\AtBeginSection[]
{
  \begin{frame}
    \frametitle{Plan}
    \tableofcontents[currentsection]
  \end{frame}
}

\begin{document}
\frame{\titlepage}
\frame{\tableofcontents}

\section{Introduction}
\begin{frame}
  \frametitle{Introduction}
\end{frame}

\section{Monde et environnement}
\begin{frame}
  
\end{frame}

\section{IA et gestion des entités}

\subsection{BDI}
\begin{frame}
  \frametitle{IA des PNJs}
  Fonctionnement des PNJs :
  \begin{itemize}
  \item Observer autour d'eux
  \item Choisir une action
  \item Effectuer cette action
  \end{itemize}
  ~\\  
  \onslide<2>{
    Implémentation sous forme de BDI : 
    \begin{itemize}
    \item Connaissances
    \item Désirs et plan d'action
    \item Actions
    \end{itemize}
  }
\end{frame}

\begin{frame}
  \frametitle{IA des PNJs}
  Les connaissances :
  \begin{itemize}
  \item Fait
  \item Position
  \item Possession
  \end{itemize}
  ~\\
  \onslide<2->{
    Les désirs sont des listes d'actions
    ~\\
    2 désirs possibles :
    \begin{itemize}
    \item Possession : ramasser des objets
    \item Connaissance : parler avec tout le monde
    \end{itemize}
    Si aucun désir réalisable : exploration du monde
  }
\end{frame}

\begin{frame}
  Les actions :
  \begin{itemize}
  \item Méthode à exécuter sur le PNJ
  \item Paramètres de la méthode
  \item Représenté en chaine de caractère
  \item Appel de la méthode par introspection
  \end{itemize}
\end{frame}

\subsection{DatabaseManager}
\begin{frame}
  \frametitle{Gestion des entités}
  Classe \textbf{DatabaseManager} : Injection de dépendances
  \begin{itemize}
  \item Stockage de l'ensemble des entités
  \item Récupération à partir d'un id
  \item Recherche à partir d'un ou plusieurs attributs
  \item Création d'objet ou recherche d'entité ``from String''
  \end{itemize}
\end{frame}

\section{Interactions utilisateur}
\begin{frame}
  
\end{frame}

\section{Interface graphique}
\begin{frame}
  
\end{frame}

\section{Demonstration}
\begin{frame}
  
\end{frame}


\section{Conclusion}
\begin{frame}
  \frametitle{Conclusion}
\end{frame}


\end{document}
